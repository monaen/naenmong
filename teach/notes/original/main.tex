%-----------------------Homework------------------------------------
%-------------------Arman Shokrollahi---------------------------------
%---------------------Introduction to Electrical and Electronic Engineering-------------------------------

\documentclass[a4 paper]{article}
% Set target color model to RGB
\usepackage[inner=1.5cm,outer=1.5cm,top=2.5cm,bottom=2.5cm]{geometry}
\usepackage{setspace}
\usepackage[rgb]{xcolor}
\usepackage{verbatim}
\usepackage{amsgen,amsmath,amstext,amsbsy,amsopn,tikz,amssymb,tkz-linknodes}
\usepackage{fancyhdr}
\usepackage[colorlinks=true, urlcolor=blue,  linkcolor=blue, citecolor=blue]{hyperref}
\usepackage[colorinlistoftodos]{todonotes}
\usepackage{rotating}
%\usetikzlibrary{through,backgrounds}
\hypersetup{%
pdfauthor={Nan Meng},%
pdftitle={Notes},%
pdfkeywords={Tikz,latex,bootstrap,uncertaintes},%
pdfcreator={PDFLaTeX},%
pdfproducer={PDFLaTeX},%
}
%\usetikzlibrary{shadows}
\usepackage[francais]{babel}
\usepackage{booktabs}
\newcommand{\ra}[1]{\renewcommand{\arraystretch}{#1}}

      \newtheorem{thm}{Theorem}[section]
      \newtheorem{prop}[thm]{Proposition}
      \newtheorem{lem}[thm]{Lemma}
      \newtheorem{cor}[thm]{Corollary}
      \newtheorem{defn}[thm]{Definition}
      \newtheorem{rem}[thm]{Remark}
      \numberwithin{equation}{section}

\newcommand{\homework}[6]{
   \pagestyle{myheadings}
   \thispagestyle{plain}
   \newpage
   \setcounter{page}{1}
   \noindent
   \begin{center}
   \framebox{
      \vbox{\vspace{2mm}
    \hbox to 6.28in { {\bf ENGG1203:~Introduction to Electrical and Electronic Engineering \hfill} }
       \vspace{6mm}
       \hbox to 6.28in { {\Large \hfill #1 (#2)  \hfill} }
       \vspace{6mm}
       \hbox to 6.28in { {\it Instructor: #3 \hfill TA: #5} }
       %\hbox to 6.28in { {\it TA: #4  \hfill #6}}
      \vspace{2mm}}
   }
   \end{center}
   \markboth{#5 -- #1}{#5 -- #1}
   \vspace*{4mm}
}

\newcommand{\bbF}{\mathbb{F}}
\newcommand{\bbX}{\mathbb{X}}
\newcommand{\bI}{\mathbf{I}}
\newcommand{\bX}{\mathbf{X}}
\newcommand{\bY}{\mathbf{Y}}
\newcommand{\bepsilon}{\boldsymbol{\epsilon}}
\newcommand{\balpha}{\boldsymbol{\alpha}}
\newcommand{\bbeta}{\boldsymbol{\beta}}
\newcommand{\0}{\mathbf{0}}

\begin{document}
\homework{Notes: Circuits}{Part 1}{Edmund Y.Lam}{}{Nan Meng}{}

{\begin{tikzpicture}[outline/.style={draw=#1,thick,fill=#1!50}]
\node [outline=red] at (0,1) {\bf Problem 1};
\end{tikzpicture}}

\textit{Problem $1.1$ of Ryan/Lin} (We have changed the received word):

A single error has been added (modulo $2$) to a transmitted $(7,4)$ Hamming codeword, resulting in the received word $r = (1100 \; 001)$.  Using the decoding algorithm described in the chapter, find the error.

\vskip 1em

{\fbox{\it Solution}}

We want to solve this problem in two ways.

\underline{First Way}:

\begin{center}
\begin{tikzpicture}[thin,fill opacity=0.7]
\draw [draw=none, fill=red] (90:1.5) circle (2cm);
\draw [draw=none, fill=green] (-30:1.5) circle (2cm);
\draw [draw=none, fill=blue] (210:1.5) circle (2cm);
\begin{scope}
	\clip (90:1.5) circle(2cm);
	\draw [draw=none, fill=yellow] (-30:1.5) circle (2cm);
\end{scope} % blue + red = magenta
\begin{scope}
	\clip (210:1.5) circle(2cm);
	\draw [draw=none, fill=magenta] (90:1.5) circle (2cm);
\end{scope}
\node at (4.1,0) {Circle 2};
\node at (-4.1,0) {Circle 3};
\begin{scope} % green + blue = cyan
	\clip (-30:1.5) circle(2cm);
	\draw [draw=none, fill=cyan] (210:1.5) circle (2cm);
\end{scope}
\node at (0,4.1) {Circle 1};
\begin{scope}
	\clip (90:1.5) circle(2cm);
	\clip (210:1.5) circle(2cm);
	\draw [draw=none, fill=white] (-30:1.5) circle (2cm);	
\end{scope}
\node at (1,.5) {$u_0$};
\node at (0,0) {$u_2$};
\node at (1.8,-1.1) {$p_1$};
\node at (-1,.5) {$u_3$};
\node at (0,-1.2) {$u_1$};
\node at (0,2) {$p_0$};
\node at (-2,-1.2) {$p_2$};
\end{tikzpicture}
\end{center}
%
We know that ${\mathbf v}=({\mathbf{u \; p}})=(u_0u_1u_2u_3 \; p_0p_1p_2)$ has been transmitted. We have
$r = (1100 \; 001)=(r_0r_1r_2r_3 \; r_4r_5r_6)$ as the received word. We rearrange the above Venn diagram as follows
%%
\begin{center}
\begin{tikzpicture}[thin,fill opacity=0.7]
\draw [draw=none, fill=red] (90:1.5) circle (2cm);
\draw [draw=none, fill=green] (-30:1.5) circle (2cm);
\draw [draw=none, fill=blue] (210:1.5) circle (2cm);
\begin{scope}
	\clip (90:1.5) circle(2cm);
	\draw [draw=none, fill=yellow] (-30:1.5) circle (2cm);
\end{scope} % blue + red = magenta
\begin{scope}
	\clip (210:1.5) circle(2cm);
	\draw [draw=none, fill=magenta] (90:1.5) circle (2cm);
\end{scope}
\node at (4.1,0) {Circle $2$};
\node at (-4.1,0) {Circle $3$};
\begin{scope} % green + blue = cyan
	\clip (-30:1.5) circle(2cm);
	\draw [draw=none, fill=cyan] (210:1.5) circle (2cm);
\end{scope}
\node at (0,4.1) {Circle $1$};
\begin{scope}
	\clip (90:1.5) circle(2cm);
	\clip (210:1.5) circle(2cm);
	\draw [draw=none, fill=white] (-30:1.5) circle (2cm);	
\end{scope}
\node at (1,.5) {$r_0=1$};
\node at (0,0) {$r_2=0$};
\node at (1.8,-1.1) {$r_5=0$};
\node at (-1,.5) {$r_3=0$};
\node at (0,-1.2) {$r_1=1$};
\node at (0,2) {$r_4=0$};
\node at (-2,-1.2) {$r_6=1$};
\end{tikzpicture}
\end{center}

Clearly, Circles $2$ and $3$, in the figure above, have even numbers of $1$'s, but Circle $1$ does not. We conclude that the error cannot be in Circles $2$ and $3$, because their rules are satisfied. So it must be $r_4=0$ that is in error. Thus, $r_4$ must be $1$. Hence, the decoded codeword is
\begin{equation}
{\mathbf{\hat{v}}}=(1100 \; 101),
\label{equ1.1}
\end{equation}
from which the decoded data ${\mathbf{\hat{u}}}=(1100)$ may be recovered. \\

%%
%%
%
\underline{Second Way}:

We want to solve the problem by some techniques based on generator and parity-check matrices. The parity-check matrix $H$ is helpful in correcting single errors in transmission when
\begin{enumerate}
  \item[(i)] $H$ has no column of $0$'s,
  \item[(ii)] no two columns of $H$ are the same.
\end{enumerate}
Consider the following matrix

$$H=\left(
\begin{array}{cccccccc}
  1 & 0 & 1 & 1 & \vdots & 1 & 0 & 0 \\
  1 & 1 & 0 & 1 & \vdots & 0 & 1 & 0 \\
  0 & 1 & 1 & 1 & \vdots & 0 & 0 & 1
\end{array}\right).$$

 It is easy to check that $H$ satisfies these two conditions and that for the number of rows ($r=3$) in $H$, we have the maximum number of columns possible. If an additional column is added, $H$ will no longer be useful for correcting single errors.\\
The generator matrix $G$ associated with $H$ is
$$G=\left(
  \begin{array}{cccccccc}
    1 & 0 & 0 & 0 &\vdots & 1 & 1 & 0 \\
    0 & 1 & 0 & 0 & \vdots& 0 & 1 & 1 \\
    0 & 0 & 1 & 0 & \vdots& 1 & 0 & 1 \\
    0 & 0 & 0 & 1 &\vdots & 1 & 1 & 1 \\
  \end{array}\right).$$
  Consequently we have a $(7,4)$ group code. The encoding function $E:\mathbb{Z}_2^4 \rightarrow \mathbb{Z}_2^7$ encodes four-bit messages into seven-bit code words. We realize that because $H$ is determined by three parity-check equations (that is, For all $w=w_1w_2w_3w_4\in\mathbb{Z}_2^4$, and $E(w)= wG = w_1w_2w_3w_4w_5w_6w_7 \in \mathbb{Z}_2^7$, now try to find $E(w)=wG$. We get some general equations which are called the {\textit{parity-check equations}}. For more details, see pages 97, 98 of Ryan/Lin), we have now maximized the number of bits we can have in the messages (of course, under our present coding scheme). In addition, the columns of $H$, read from top to bottom, are the binary equivalents of the integers from $1$ to $7$. In general, if we start with $r$ parity-check equations, then the parity-check matrix $H$ can have as many as $2^r-1$ columns and still be used to correct single errors. We denote the transposition of $B$ by $B^{tr}$. Under these circumstances $H=[B\ | \ I_r]$, where $B$ is an $r\times (2^r-1-r)$ matrix, and $G=[I_m\ | \ B^{tr}]$ with $m=2^r-1-r$. The parity-check matrix $H$ associated with a $(2^r-1, 2^r-1-r)$ group code. \\
  We want to use some terminologies which can be found on pages 103, 104, and 105 of Ryan/Lin. We now have the matrix $H$ for a Hamming $(7,4)$ code. It is easy to check that the coset leader for the syndrome $(100)$ is $(0000 \; 100)$. Why we are talking about  $(100)$? because it is the syndrome corresponding to our received word $r=(1100 \; 001)$; note that
  $$H \cdot r^{tr}=
  \left(
    \begin{array}{c}
      1 \\
      0 \\
      0 \\
    \end{array}
  \right).$$
  Finally, if we assume that $c$ is the transmitted word, then $c=(0000 \; 100)+(1100 \; 001)= (1100 \; 101)\stackrel{\text{\tiny\ref{equ1.1}}}{=} \bf\hat{v}$. \\


%
%
%
%
%
\end{document}
%%------------ Arman Shokrollahi--------------%%
