%%%%%%%%%%%%%%%%%%%%%%%%%%%%%%%%%%%%%%%%%
% Beamer Presentation
% LaTeX Template
% Version 1.0 (10/11/12)
%
% This template has been downloaded from:
% http://www.LaTeXTemplates.com
%
% License:
% CC BY-NC-SA 3.0 (http://creativecommons.org/licenses/by-nc-sa/3.0/)
%
%%%%%%%%%%%%%%%%%%%%%%%%%%%%%%%%%%%%%%%%%

%----------------------------------------------------------------------------------------
%	PACKAGES AND THEMES
%----------------------------------------------------------------------------------------

\documentclass[xcolor=table]{beamer}

\mode<presentation> {


%\usetheme{Berlin}
%\usetheme{Boadilla}
\usetheme{CambridgeUS}
%\usetheme{Copenhagen}
%\usetheme{Darmstadt}
%\usetheme{Dresden}

%\usecolortheme{wolverine}

%\setbeamertemplate{footline} % To remove the footer line in all slides uncomment this line
%\setbeamertemplate{footline}[page number] % To replace the footer line in all slides with a simple slide count uncomment this line

%\setbeamertemplate{navigation symbols}{} % To remove the navigation symbols from the bottom of all slides uncomment this line
}

\usefonttheme{professionalfonts}
\usepackage{times}
\usepackage{tikz}
\usepackage{amsmath}
\usepackage{verbatim}
\usetikzlibrary{arrows,shapes}
\usepackage{float}
\usepackage{graphicx}
\usepackage[outdir=./]{epstopdf}
\usepackage{booktabs}
\usepackage{multirow}
\usepackage[english]{babel}
\usepackage[utf8]{inputenc}
\usepackage{algorithm}
\usepackage[noend]{algpseudocode}
\usepackage{hyperref}
\usepackage{fmtcount} % equivalent to \usepackage[super]{nth}
\usepackage{listings}
\usepackage{color}
\newcommand{\blue}[1]{\textcolor{blue}{#1}}
\newcommand{\red}[1]{\textcolor{red}{#1}}
\newcommand{\green}[1]{\textcolor{green}{#1}}
\newcommand{\yellow}[1]{\textcolor{yellow}{#1}}
\newcommand{\orange}[1]{\textcolor{orange}{#1}}
\newcommand{\cyan}[1]{\textcolor{cyan}{#1}}
\newcommand{\magenta}[1]{\textcolor{magenta}{#1}}

%----------------------------------------------------------------------------------------
%	TITLE PAGE
%----------------------------------------------------------------------------------------

\title[Two-party Jointly Decision Tree]{Privacy Preserving Distributed ID3 Algorithm} % The short title appears at the bottom of every slide, the full title is only on the title page

\author{Chen Yuechen \\ Fung Divine \\ Nan Meng} % Your name

\date{\today} % Date, can be changed to a custom date

\begin{document}

\begin{frame}
	\titlepage % Print the title page as the first slide
\end{frame}

\begin{frame}
\frametitle{Overview} % Table of contents slide, comment this block out to remove it

\begin{itemize} \itemsep16pt \parskip0pt \parsep0pt
  \item[$\ast$] Introduction
  \item[$\ast$] Problem Definition
  \item[$\ast$] Solution
  \item[$\ast$] Result
  \item[$\ast$] Conclusion
\end{itemize}
\end{frame}

%----------------------------------------------------------------------------------------
%	PRESENTATION SLIDES
%----------------------------------------------------------------------------------------
%------------------------------------------------
\begin{frame}
\frametitle{Privacy Preserving Data Mining}

\begin{itemize} \itemsep16pt \parskip0pt \parsep0pt
  \item[$\bullet$] Mining while protecting the privacy of data.
\end{itemize}

\begin{columns}
\begin{column}{6cm}
\begin{figure}[H]
  \caption{Lindell’s definition}
  \centering
  \includegraphics[width=0.8\textwidth]{presentation1}
  \label{Angelinatree}
\end{figure}
\end{column}


\begin{column}{6cm}
\begin{figure}[H]
  \caption{Agrawal’s definition}
  \centering
  \includegraphics[width=0.55\textwidth]{presentation2}
  \label{Angelinatree}
\end{figure}
\end{column}
\end{columns}
\end{frame}

%------------------------------------------------
%------------------------------------------------
\begin{frame}
\frametitle{ID3 Algorithm}
\begin{itemize} \itemsep16pt \parskip0pt \parsep0pt
  \item[$\bullet$] ID3 is an algorithm used to generate a decision tree from a dataset, and is typically used in the data mining.
  \begin{itemize} \itemsep10pt \parskip0pt \parsep0pt
    \item[$1.$] Calculate the {\bf entropy} of every attribute using the data set S.
    \item[$2.$] Split the set S into subsets using the attribute for which entropy is minimum 
    \item[$3.$] Make a decision tree node containing that attribute
    \item[$4.$] Recurse on subsets using remaining attributes.
  \end{itemize}
\end{itemize}

\end{frame}

%------------------------------------------------
%------------------------------------------------
\begin{frame}
\frametitle{Distributed ID3 Algorithm}
\begin{columns}

\begin{column}{8cm}
\begin{table}[]
\centering
\caption{Play Golf Dataset}
\begin{tabular}{lllll}
\rowcolor[HTML]{9B9B9B} 
Outlook  & Temp & Humidity & Windy & Play Golf \\
\rowcolor[HTML]{FFCB2F} 
Rainy    & Hot  & High     & FALSE & No        \\
\rowcolor[HTML]{FFCB2F} 
Rainy    & Hot  & High     & TRUE  & No        \\
\rowcolor[HTML]{FFCB2F} 
Overcast & Hot  & High     & FALSE & Yes       \\
\rowcolor[HTML]{FFCB2F} 
Sunny    & Mild & High     & FALSE & Yes       \\
\rowcolor[HTML]{FFCB2F} 
Rainy    & Mild & Normal   & TRUE  & Yes       \\
\rowcolor[HTML]{34CDF9} 
Overcast & Cool & Normal   & TRUE  & Yes       \\
\rowcolor[HTML]{34CDF9} 
Rainy    & Mild & High     & FALSE & No        \\
\rowcolor[HTML]{34CDF9} 
Rainy    & Cool & Normal   & FALSE & Yes       \\
\rowcolor[HTML]{34CDF9} 
Sunny    & Mild & Normal   & FALSE & Yes      
\end{tabular}
\end{table}
\end{column}



\begin{column}{2cm}
\vspace{2cm} \newline
{\large $\Rightarrow$} \hspace{4mm} \red{\bf Alice} \newline
\vspace{1cm} \newline
{\large $\Rightarrow$} \hspace{4mm} \cyan{\bf Bob}
\end{column}

\end{columns}

\end{frame}

%------------------------------------------------
%------------------------------------------------
\begin{frame}
\frametitle{Distributed ID3 Algorithm}
\begin{itemize} \itemsep16pt \parskip0pt \parsep0pt
  \item[$\bullet$] Data is distributed in two or more parties
\end{itemize}

\begin{columns}

\begin{column}{6cm}
\begin{table}[]
\centering
\caption{Alice}
\scalebox{0.6}{
\begin{tabular}{lllll}
\rowcolor[HTML]{9B9B9B} 
Outlook  & Temp & Humidity & Windy & Play Golf \\
\rowcolor[HTML]{FFCB2F} 
Rainy    & Hot  & High     & FALSE & No        \\
\rowcolor[HTML]{FFCB2F} 
Rainy    & Hot  & High     & TRUE  & No        \\
\rowcolor[HTML]{FFCB2F} 
Overcast & Hot  & High     & FALSE & Yes       \\
\rowcolor[HTML]{FFCB2F} 
Sunny    & Mild & High     & FALSE & Yes       \\
\rowcolor[HTML]{FFCB2F} 
Rainy    & Mild & Normal   & TRUE  & Yes      
\end{tabular}}
\end{table}

\end{column}

\begin{column}{6cm}

\begin{table}[]
\centering
\caption{Bob}
\scalebox{0.6}{
\begin{tabular}{lllll}
\rowcolor[HTML]{9B9B9B} 
Outlook  & Temp & Humidity & Windy & Play Golf \\
\rowcolor[HTML]{34CDF9} 
Overcast & Cool & Normal   & TRUE  & Yes       \\
\rowcolor[HTML]{34CDF9} 
Rainy    & Mild & High     & FALSE & No        \\
\rowcolor[HTML]{34CDF9} 
Rainy    & Cool & Normal   & FALSE & Yes       \\
\rowcolor[HTML]{34CDF9} 
Sunny    & Mild & Normal   & FALSE & Yes      
\end{tabular}}
\end{table}
\vspace{4 mm}
\end{column}
\end{columns}

\begin{itemize} \itemsep16pt \parskip0pt \parsep0pt
  \item[$\bullet$] Combine data together and get a decision tree
\end{itemize}

\end{frame}

%------------------------------------------------
%------------------------------------------------
\begin{frame}
\frametitle{Problem Definition}

\begin{itemize} \itemsep16pt \parskip0pt \parsep0pt
  \item[$\bullet$]However, data is \red{\bf privacy}.
\end{itemize}

\begin{columns}

\begin{column}{6cm}
\begin{table}[]
\centering
\caption{Alice}
\scalebox{0.6}{
\begin{tabular}{lllll}
\rowcolor[HTML]{9B9B9B} 
Outlook  & Temp & Humidity & Windy & Play Golf \\
\rowcolor[HTML]{FFCB2F} 
Rainy    & Hot  & High     & FALSE & No        \\
\rowcolor[HTML]{FFCB2F} 
Rainy    & Hot  & High     & TRUE  & No        \\
\rowcolor[HTML]{FFCB2F} 
Overcast & Hot  & High     & FALSE & Yes       \\
\rowcolor[HTML]{FFCB2F} 
Sunny    & Mild & High     & FALSE & Yes       \\
\rowcolor[HTML]{FFCB2F} 
Rainy    & Mild & Normal   & TRUE  & Yes      
\end{tabular}}
\end{table}

\end{column}

\begin{column}{6cm}

\begin{table}[]
\centering
\caption{Bob}
\scalebox{0.6}{
\begin{tabular}{lllll}
\rowcolor[HTML]{9B9B9B} 
Outlook  & Temp & Humidity & Windy & Play Golf \\
\rowcolor[HTML]{34CDF9} 
Overcast & Cool & Normal   & TRUE  & Yes       \\
\rowcolor[HTML]{34CDF9} 
Rainy    & Mild & High     & FALSE & No        \\
\rowcolor[HTML]{34CDF9} 
Rainy    & Cool & Normal   & FALSE & Yes       \\
\rowcolor[HTML]{34CDF9} 
Sunny    & Mild & Normal   & FALSE & Yes      
\end{tabular}}
\end{table}
\vspace{4 mm}
\end{column}
\end{columns}

\begin{itemize} \itemsep16pt \parskip0pt \parsep0pt
  \item[$\bullet$] How to share data in a safe way in distributed ID3 algorithm?
\end{itemize}
\end{frame}

%------------------------------------------------
%------------------------------------------------
\begin{frame}
\frametitle{An Example of Distributed ID3 Algorithm}
\begin{itemize} \itemsep16pt \parskip0pt \parsep0pt
  \item[$\bullet$] Here we use a example of Distributed ID3 algorithm to clearly define the problem. For example, Compute the entropy of \red{\bf Rainy}.
\end{itemize}
\begin{columns}
\begin{column}{6cm}
\begin{table}[]
\centering
\caption{Alice}
\scalebox{0.6}{
\begin{tabular}{lllll}
\rowcolor[HTML]{9B9B9B} 
Outlook  & Temp & Humidity & Windy & Play Golf \\
\rowcolor[HTML]{FFCB2F} 
\red{Rainy}    & Hot  & High     & FALSE & \red{No}        \\
\rowcolor[HTML]{FFCB2F} 
\red{Rainy}    & Hot  & High     & TRUE  & \red{No}        \\
\rowcolor[HTML]{FFCB2F} 
Overcast & Hot  & High     & FALSE & Yes       \\
\rowcolor[HTML]{FFCB2F} 
Sunny    & Mild & High     & FALSE & Yes       \\
\rowcolor[HTML]{FFCB2F} 
\red{Rainy}    & Mild & Normal   & TRUE  & \red{Yes}      
\end{tabular}}
\end{table}
\centerline{\red{3} records, \red{2} No, \red{1} Yes}
\end{column}

\begin{column}{6cm}

\begin{table}[]
\centering
\caption{My caption}
\scalebox{0.6}{
\begin{tabular}{lllll}
\rowcolor[HTML]{9B9B9B} 
Outlook  & Temp & Humidity & Windy & Play Golf \\
\rowcolor[HTML]{34CDF9} 
Overcast & Cool & Normal   & TRUE  & Yes       \\
\rowcolor[HTML]{34CDF9} 
\red{Rainy}    & Mild & High     & FALSE & \red{No}        \\
\rowcolor[HTML]{34CDF9} 
\red{Rainy}    & Cool & Normal   & FALSE & \red{Yes}       \\
\rowcolor[HTML]{34CDF9} 
Sunny    & Mild & Normal   & FALSE & Yes      
\end{tabular}}
\end{table}
\vspace{3 mm}
\centerline{\cyan{2} records, \cyan{1} No, \cyan{1} Yes}
\vspace{1 mm}
\end{column}
\end{columns}

\begin{itemize} \itemsep1pt \parskip0pt \parsep0pt
  \item[] {\scriptsize \begin{equation*}Entropy(Rainy) = -\underset{Play Golf = No}{\underbrace{\frac{\color{red}{2}\color{black}{+}\color{cyan}{1}}{\color{red}{3}\color{black}{+}\color{cyan}{2}}log_2{(\frac{\color{red}{2}\color{black}{+}\color{cyan}{1}}{\color{red}{3}\color{black}{+}\color{cyan}{2}}})}} - \underset{PlayGolf = Yes}{\underbrace{\frac{\color{red}{1}\color{black}{+}\color{cyan}{1}}{\color{red}{3}\color{black}{+}\color{cyan}{2}}log_2{(\frac{\color{red}{1}\color{black}{+}\color{cyan}{1}}{\color{red}{3}\color{black}{+}\color{cyan}{2}}})}}\end{equation*}}
  \item[] \hspace{43mm}{\scriptsize $ = -\frac{3}{5}log_2{(\frac{3}{5})} - \frac{2}{5}log_2{(\frac{2}{5})}$}
\end{itemize}

\end{frame}

%------------------------------------------------
%------------------------------------------------
\begin{frame}
\frametitle{An Example of Distributed ID3 Algorithm}
\begin{itemize} \itemsep16pt \parskip0pt \parsep0pt
  \item[$\bullet$] For example, Compute the entropy of \red{\bf Rainy}.
\end{itemize}

\begin{columns}

\begin{column}{6cm}
\begin{table}[]
\centering
\caption{Alice}
\scalebox{0.6}{
\begin{tabular}{lllll}
\rowcolor[HTML]{9B9B9B} 
Outlook  & Temp & Humidity & Windy & Play Golf \\
\rowcolor[HTML]{FFCB2F} 
\red{Rainy}    & Hot  & High     & FALSE & \red{No}        \\
\rowcolor[HTML]{FFCB2F} 
\red{Rainy}    & Hot  & High     & TRUE  & \red{No}        \\
\rowcolor[HTML]{FFCB2F} 
Overcast & Hot  & High     & FALSE & Yes       \\
\rowcolor[HTML]{FFCB2F} 
Sunny    & Mild & High     & FALSE & Yes       \\
\rowcolor[HTML]{FFCB2F} 
\red{Rainy}    & Mild & Normal   & TRUE  & \red{Yes}      
\end{tabular}}
\end{table}
\centerline{\red{3} records, \red{2} No, 1 Yes}
\end{column}

\begin{column}{6cm}

\begin{table}[]
\centering
\caption{My caption}
\scalebox{0.6}{
\begin{tabular}{lllll}
\rowcolor[HTML]{9B9B9B} 
Outlook  & Temp & Humidity & Windy & Play Golf \\
\rowcolor[HTML]{34CDF9} 
Overcast & Cool & Normal   & TRUE  & Yes       \\
\rowcolor[HTML]{34CDF9} 
\red{Rainy}    & Mild & High     & FALSE & \red{No}        \\
\rowcolor[HTML]{34CDF9} 
\red{Rainy}    & Cool & Normal   & FALSE & \red{Yes}       \\
\rowcolor[HTML]{34CDF9} 
Sunny    & Mild & Normal   & FALSE & Yes      
\end{tabular}}
\end{table}
\vspace{3 mm}
\centerline{\cyan{2} records, \cyan{1} No, 1 Yes}
\vspace{1 mm}
\end{column}
\end{columns}





% Below we mix an ordinary equation with TikZ nodes. Note that we have to
% adjust the baseline of the nodes to get proper alignment with the rest of
% the equation.
\begin{itemize} \itemsep1pt \parskip0pt \parsep0pt
  \item[] {\scriptsize \begin{equation*}
Entropy(Rainy) = 
        \tikz[baseline]{
            \node[fill=cyan!20,anchor=base]
            {$ -\frac{\color{red}{2}\color{black}{+}\color{cyan}{1}}{\color{red}{3}\color{black}{+}\color{cyan}{2}}log_2{(\frac{\color{red}{2}\color{black}{+}\color{cyan}{1}}{\color{red}{3}\color{black}{+}\color{cyan}{2}}})$};
        } -
        \tikz[baseline]{
            \node[anchor=base]
            {$\frac{\color{red}{1}\color{black}{+}\color{cyan}{1}}{\color{red}{3}\color{black}{+}\color{cyan}{2}}log_2{(\frac{\color{red}{1}\color{black}{+}\color{cyan}{1}}{\color{red}{3}\color{black}{+}\color{cyan}{2}}})$};
        }
\end{equation*}}
  \item[] \hspace{42mm} {\scriptsize$ = -\frac{3}{5}log_2{(\frac{3}{5})} - \frac{2}{5}log_2{(\frac{2}{5})}$}
\end{itemize}

\end{frame}

%------------------------------------------------
%------------------------------------------------
\begin{frame}
\frametitle{An Example of Distributed ID3 Algorithm}
\begin{itemize} \itemsep16pt \parskip0pt \parsep0pt
  \item[$\bullet$] For example, Compute the entropy of \red{\bf Rainy}.
\end{itemize}

\begin{columns}

\begin{column}{6cm}
\begin{table}[]
\centering
\caption{Alice}
\scalebox{0.6}{
\begin{tabular}{lllll}
\rowcolor[HTML]{9B9B9B} 
Outlook  & Temp & Humidity & Windy & Play Golf \\
\rowcolor[HTML]{FFCB2F} 
\red{Rainy}    & Hot  & High     & FALSE & \red{No}        \\
\rowcolor[HTML]{FFCB2F} 
\red{Rainy}    & Hot  & High     & TRUE  & \red{No}        \\
\rowcolor[HTML]{FFCB2F} 
Overcast & Hot  & High     & FALSE & Yes       \\
\rowcolor[HTML]{FFCB2F} 
Sunny    & Mild & High     & FALSE & Yes       \\
\rowcolor[HTML]{FFCB2F} 
\red{Rainy}    & Mild & Normal   & TRUE  & \red{Yes}      
\end{tabular}}
\end{table}
\centerline{\red{3} records, \red{2} No, 1 Yes}
\end{column}

\begin{column}{6cm}

\begin{table}[]
\centering
\caption{My caption}
\scalebox{0.6}{
\begin{tabular}{lllll}
\rowcolor[HTML]{9B9B9B} 
Outlook  & Temp & Humidity & Windy & Play Golf \\
\rowcolor[HTML]{34CDF9} 
Overcast & Cool & Normal   & TRUE  & Yes       \\
\rowcolor[HTML]{34CDF9} 
\red{Rainy}    & Mild & High     & FALSE & \red{No}        \\
\rowcolor[HTML]{34CDF9} 
\red{Rainy}    & Cool & Normal   & FALSE & \red{Yes}       \\
\rowcolor[HTML]{34CDF9} 
Sunny    & Mild & Normal   & FALSE & Yes      
\end{tabular}}
\end{table}
\vspace{3 mm}
\centerline{\cyan{2} records, \cyan{1} No, 1 Yes}
\vspace{1 mm}
\end{column}
\end{columns}





% Below we mix an ordinary equation with TikZ nodes. Note that we have to
% adjust the baseline of the nodes to get proper alignment with the rest of
% the equation.
\begin{itemize} \itemsep1pt \parskip0pt \parsep0pt
  \item[] {\scriptsize \begin{equation*}
        \tikz[baseline]{
            \node[fill=cyan!20,anchor=base]
            {$ -\frac{\color{red}{2}\color{black}{+}\color{cyan}{1}}{\color{red}{3}\color{black}{+}\color{cyan}{2}}log_2{(\frac{\color{red}{2}\color{black}{+}\color{cyan}{1}}{\color{red}{3}\color{black}{+}\color{cyan}{2}}})$};
        }
\end{equation*}}
  
\end{itemize}

\end{frame}

%------------------------------------------------
%------------------------------------------------
\begin{frame}
\frametitle{An Example of Distributed ID3 Algorithm}
\begin{itemize} \itemsep16pt \parskip0pt \parsep0pt
  \item[$\bullet$] For example, Compute the entropy of \red{\bf Rainy}.
\end{itemize}

\begin{columns}

\begin{column}{6cm}
\begin{table}[]
\centering
\caption{Alice}
\scalebox{0.6}{
\begin{tabular}{lllll}
\rowcolor[HTML]{9B9B9B} 
Outlook  & Temp & Humidity & Windy & Play Golf \\
\rowcolor[HTML]{FFCB2F} 
\red{Rainy}    & Hot  & High     & FALSE & \red{No}        \\
\rowcolor[HTML]{FFCB2F} 
\red{Rainy}    & Hot  & High     & TRUE  & \red{No}        \\
\rowcolor[HTML]{FFCB2F} 
Overcast & Hot  & High     & FALSE & Yes       \\
\rowcolor[HTML]{FFCB2F} 
Sunny    & Mild & High     & FALSE & Yes       \\
\rowcolor[HTML]{FFCB2F} 
\red{Rainy}    & Mild & Normal   & TRUE  & \red{Yes}      
\end{tabular}}
\end{table}
\centerline{\red{3} records, \red{2} No, 1 Yes}
\end{column}

\begin{column}{6cm}

\begin{table}[]
\centering
\caption{My caption}
\scalebox{0.6}{
\begin{tabular}{lllll}
\rowcolor[HTML]{9B9B9B} 
Outlook  & Temp & Humidity & Windy & Play Golf \\
\rowcolor[HTML]{34CDF9} 
Overcast & Cool & Normal   & TRUE  & Yes       \\
\rowcolor[HTML]{34CDF9} 
\red{Rainy}    & Mild & High     & FALSE & \red{No}        \\
\rowcolor[HTML]{34CDF9} 
\red{Rainy}    & Cool & Normal   & FALSE & \red{Yes}       \\
\rowcolor[HTML]{34CDF9} 
Sunny    & Mild & Normal   & FALSE & Yes      
\end{tabular}}
\end{table}
\vspace{3 mm}
\centerline{\cyan{2} records, \cyan{1} No, 1 Yes}
\vspace{1 mm}
\end{column}
\end{columns}





% Below we mix an ordinary equation with TikZ nodes. Note that we have to
% adjust the baseline of the nodes to get proper alignment with the rest of
% the equation.
\begin{itemize} \itemsep1pt \parskip0pt \parsep0pt
  \item[]  \begin{equation*}
  -\frac{\color{red}{2}\color{black}{+}\color{cyan}{1}}{\color{red}{3}\color{black}{+}\color{cyan}{2}}log_2
        \tikz[baseline]{
            \node[fill=cyan!20,anchor=base]
            {$ (\frac{\color{red}{2}\color{black}{+}\color{cyan}{1}}{\color{red}{3}\color{black}{+}\color{cyan}{2}})$};
        }
\end{equation*}
  
\end{itemize}

\end{frame}

%------------------------------------------------
%------------------------------------------------
\begin{frame}
\frametitle{An Example of Distributed ID3 Algorithm}
\begin{itemize} \itemsep16pt \parskip0pt \parsep0pt
  \item[$\bullet$] For example, Compute the entropy of \red{\bf Rainy}.
\end{itemize}

\begin{columns}

\begin{column}{6cm}
\begin{table}[]
\centering
\caption{Alice}
\scalebox{0.6}{
\begin{tabular}{lllll}
\rowcolor[HTML]{9B9B9B} 
Outlook  & Temp & Humidity & Windy & Play Golf \\
\rowcolor[HTML]{FFCB2F} 
\red{Rainy}    & Hot  & High     & FALSE & \red{No}        \\
\rowcolor[HTML]{FFCB2F} 
\red{Rainy}    & Hot  & High     & TRUE  & \red{No}        \\
\rowcolor[HTML]{FFCB2F} 
Overcast & Hot  & High     & FALSE & Yes       \\
\rowcolor[HTML]{FFCB2F} 
Sunny    & Mild & High     & FALSE & Yes       \\
\rowcolor[HTML]{FFCB2F} 
\red{Rainy}    & Mild & Normal   & TRUE  & \red{Yes}      
\end{tabular}}
\end{table}
\centerline{\red{3} records, \red{2} No, 1 Yes}
\end{column}

\begin{column}{6cm}

\begin{table}[]
\centering
\caption{My caption}
\scalebox{0.6}{
\begin{tabular}{lllll}
\rowcolor[HTML]{9B9B9B} 
Outlook  & Temp & Humidity & Windy & Play Golf \\
\rowcolor[HTML]{34CDF9} 
Overcast & Cool & Normal   & TRUE  & Yes       \\
\rowcolor[HTML]{34CDF9} 
\red{Rainy}    & Mild & High     & FALSE & \red{No}        \\
\rowcolor[HTML]{34CDF9} 
\red{Rainy}    & Cool & Normal   & FALSE & \red{Yes}       \\
\rowcolor[HTML]{34CDF9} 
Sunny    & Mild & Normal   & FALSE & Yes      
\end{tabular}}
\end{table}
\vspace{3 mm}
\centerline{\cyan{2} records, \cyan{1} No, 1 Yes}
\vspace{1 mm}
\end{column}
\end{columns}
% Below we mix an ordinary equation with TikZ nodes. Note that we have to
% adjust the baseline of the nodes to get proper alignment with the rest of
% the equation.
\begin{itemize} \itemsep1pt \parskip0pt \parsep0pt
  \item[] {\normalsize \begin{equation*}
        \tikz[baseline]{
            \node[anchor=base]
            {$ \frac{\color{red}{2}\color{black}{+}\color{cyan}{1}}{\color{red}{3}\color{black}{+}\color{cyan}{2}}$};
        }
\end{equation*}}
  
\end{itemize}

\end{frame}

%------------------------------------------------
%------------------------------------------------
\begin{frame}
\frametitle{An Example of Distributed ID3 Algorithm}
\begin{itemize} \itemsep16pt \parskip0pt \parsep0pt
  \item[$\bullet$] For example, Compute the entropy of \red{\bf Rainy}.
\end{itemize}

\begin{columns}

\begin{column}{6cm}
\begin{table}[]
\centering
\caption{Alice}
\scalebox{0.6}{
\begin{tabular}{lllll}
\rowcolor[HTML]{9B9B9B} 
Outlook  & Temp & Humidity & Windy & Play Golf \\
\rowcolor[HTML]{FFCB2F} 
\red{Rainy}    & Hot  & High     & FALSE & \red{No}        \\
\rowcolor[HTML]{FFCB2F} 
\red{Rainy}    & Hot  & High     & TRUE  & \red{No}        \\
\rowcolor[HTML]{FFCB2F} 
Overcast & Hot  & High     & FALSE & Yes       \\
\rowcolor[HTML]{FFCB2F} 
Sunny    & Mild & High     & FALSE & Yes       \\
\rowcolor[HTML]{FFCB2F} 
\red{Rainy}    & Mild & Normal   & TRUE  & \red{Yes}      
\end{tabular}}
\end{table}
\centerline{\red{a} records, \red{x} No, 1 Yes}
\end{column}

\begin{column}{6cm}

\begin{table}[]
\centering
\caption{My caption}
\scalebox{0.6}{
\begin{tabular}{lllll}
\rowcolor[HTML]{9B9B9B} 
Outlook  & Temp & Humidity & Windy & Play Golf \\
\rowcolor[HTML]{34CDF9} 
Overcast & Cool & Normal   & TRUE  & Yes       \\
\rowcolor[HTML]{34CDF9} 
\red{Rainy}    & Mild & High     & FALSE & \red{No}        \\
\rowcolor[HTML]{34CDF9} 
\red{Rainy}    & Cool & Normal   & FALSE & \red{Yes}       \\
\rowcolor[HTML]{34CDF9} 
Sunny    & Mild & Normal   & FALSE & Yes      
\end{tabular}}
\end{table}
\vspace{3 mm}
\centerline{\cyan{b} records, \cyan{y} No, 1 Yes}
\vspace{1 mm}
\end{column}
\end{columns}
% Below we mix an ordinary equation with TikZ nodes. Note that we have to
% adjust the baseline of the nodes to get proper alignment with the rest of
% the equation.
\begin{itemize} \itemsep1pt \parskip0pt \parsep0pt
  \item[] {\normalsize \begin{equation*}
        \tikz[baseline]{
            \node[anchor=base]
            {$ \frac{\color{red}{x}\color{black}{+}\color{cyan}{y}}{\color{red}{a}\color{black}{+}\color{cyan}{b}}$};
        }
\end{equation*}}
  
\end{itemize}

\end{frame}

%------------------------------------------------
%------------------------------------------------
\begin{frame}
\frametitle{Problem Definition}
\begin{itemize} \itemsep6pt \parskip0pt \parsep0pt
  \item[$\bullet$] Compute $\frac{\color{red}{x}\color{black}{+}\color{cyan}{y}}{\color{red}{a}\color{black}{+}\color{cyan}{b}}$ without reveal \red{a}, \red{x}, \cyan{b}, \cyan{y}.
  \item[$\bullet$] Realize Privacy Preserving Distributed ID3 algorithm.
  
\end{itemize}

\begin{figure}[H]
  \centering
  \includegraphics[width=0.5\textwidth,height=0.3\textwidth]{./presentation3}
\end{figure}



\centerline{\red{$Enc(\cdot)$ -- Encryption Algorithm}}

\end{frame}

%------------------------------------------------
%------------------------------------------------
\begin{frame}
\frametitle{Solution}


\begin{block}{PPWAP}
\vspace{4 mm}
\begin{itemize} \itemsep6pt \parskip0pt \parsep0pt
  \item[$\bullet$] \red{PPWAP}: Privacy Preserving Weight Average Protocol
\end{itemize}

\end{block}


\begin{block}{}
\begin{itemize} \itemsep6pt \parskip0pt \parsep0pt
  \item[$\bullet$] In this project, we choose PPWAP by \red{Pailier Encryption}.
\end{itemize}

\end{block}
\end{frame}

%------------------------------------------------
%------------------------------------------------
\begin{frame}
\frametitle{Pailier Encryption}

\begin{itemize} \itemsep25pt \parskip0pt \parsep0pt
  \item[$\bullet$] $KeyGeneration()$: Generate public key $PK$, and secret key $SK$.
  \item[$\bullet$] $Encryption(m, PK)$: Using $PK$ to encrypt message $m$, output $Enc(m)$.
  \item[$\bullet$] $Decryption(Enc(m), SK)$: Using SK to decrypt $Enc(m)$, output $m$.
\end{itemize}

\end{frame}

%------------------------------------------------
%------------------------------------------------
\begin{frame}
\frametitle{Pailier Encryption}

\begin{itemize} \itemsep20pt \parskip0pt \parsep0pt
  \item[$\bullet$] \red{Property}: Addition Homomorphism
  \item[$\bullet$] Given two messages \red{$m1$} and \red{$m2$}, $Enc(\color{red}{m1+m2}\color{black}{)} = Enc(\color{red}{m1}\color{black}{)} \cdot Enc(\color{red}{m2}\color{black}{)}$.
  \item[$\bullet$] The encryption of \red{$m1+m2$} can be computerd by $Enc(\color{red}{m1}\color{black}{)}$ and $Enc(\color{red}{m2}\color{black}{)}$.
\end{itemize}

\end{frame}

%------------------------------------------------
%------------------------------------------------
\begin{frame}
\frametitle{PPWAP based on Pailier Encryption}
Privacy Preserving Weighted Average Protocol
\begin{itemize} \itemsep20pt \parskip0pt \parsep0pt
  \item[$\bullet$] Within the help of Paillier, build PPWAP scheme. 
\end{itemize}
\vspace{4 mm}

\begin{figure}[H]
  \centering
  \includegraphics[width=0.5\textwidth]{./presentation4}
\end{figure}



\end{frame}

%------------------------------------------------
%------------------------------------------------
\begin{frame}
\frametitle{PPWAP}
\begin{columns}

\begin{column}{6cm}
\begin{itemize} \itemsep1pt \parskip0pt \parsep0pt
  \item[] \hspace{16mm}{Alice} \newline
  \item[$1.$] $KeyGeneration(): SK, PK$
  \item[] $Encryption(a ,PK): Enc(a)$
  \item[] $Encryption(x ,PK): Enc(x)$
\end{itemize}
\noindent\rule{12cm}{0.4pt}
\begin{itemize} \itemsep1pt \parskip0pt \parsep0pt
  \item[$2.$] 
\end{itemize}
\end{column}
\begin{column}{1cm}
\begin{itemize} \itemsep1pt \parskip0pt \parsep0pt
  \item[] \red{\Huge $\Rightarrow $}
\end{itemize}
\vspace{1 cm}
\end{column}
\begin{column}{4cm}
\begin{itemize} \itemsep1pt \parskip0pt \parsep0pt
  \item[] 
  \item[] Bob
  \item[] 
  \item[] $Enc(a)$
  \item[] $Enc(x)$
\end{itemize}
\vspace{6mm}
\begin{itemize} \itemsep1pt \parskip0pt \parsep0pt
  \item[] 
  \item[] Random integer $z$
  \item[] $Enc(a)^z, Enc(x)^z$
\end{itemize}
\end{column}
\end{columns}

\end{frame}

%------------------------------------------------
%------------------------------------------------
\begin{frame}
\frametitle{PPWAP}
\frametitle{PPWAP}
\begin{columns}

\begin{column}{6cm}
\begin{itemize} \itemsep1pt \parskip0pt \parsep0pt
  \item[] \hspace{16mm}{Alice} \newline
  \item[$1.$] $KeyGeneration(): SK, PK$
  \item[] $Encryption(a ,PK): Enc(a)$
  \item[] $Encryption(x ,PK): Enc(x)$
\end{itemize}
\noindent\rule{12cm}{0.4pt}
\begin{itemize} \itemsep1pt \parskip0pt \parsep0pt
  \item[$2.$] 
\end{itemize}
\end{column}
\begin{column}{1cm}
\begin{itemize} \itemsep1pt \parskip0pt \parsep0pt
  \item[] \red{\Huge $\Rightarrow $}
\end{itemize}
\vspace{1 cm}
\end{column}
\begin{column}{4cm}
\begin{itemize} \itemsep1pt \parskip0pt \parsep0pt
  \item[] 
  \item[] Bob
  \item[] 
  \item[] $Enc(a)$
  \item[] $Enc(x)$
\end{itemize}
\vspace{6mm}
\begin{itemize} \itemsep1pt \parskip0pt \parsep0pt
  \item[] 
  \item[] Random integer $z$
  \item[] $Enc(a)^z, Enc(x)^z$
\end{itemize}
\end{column}
\end{columns}

\begin{figure}[H]
  \centering
  \includegraphics[width=0.9\textwidth]{./presentation5}
\end{figure}




\end{frame}

%------------------------------------------------
%------------------------------------------------
\begin{frame}
\frametitle{PPWAP}
\begin{columns}

\begin{column}{6cm}
\begin{itemize} \itemsep1pt \parskip0pt \parsep0pt
  \item[] \hspace{16mm}{Alice} \newline
  \item[$1.$] $KeyGeneration(): SK, PK$
  \item[] $Encryption(a ,PK): Enc(a)$
  \item[] $Encryption(x ,PK): Enc(x)$
\end{itemize}
\noindent\rule{12cm}{0.4pt}
\begin{itemize} \itemsep1pt \parskip0pt \parsep0pt
  \item[$2.$] 
\end{itemize}
\end{column}
\begin{column}{1cm}
\begin{itemize} \itemsep1pt \parskip0pt \parsep0pt
  \item[] \red{\Huge $\Rightarrow $}
\end{itemize}
\vspace{1 cm}
\end{column}
\begin{column}{4cm}

\begin{itemize} \itemsep1pt \parskip0pt \parsep0pt
  \item[] 
  \item[] 
  \item[] Bob
  \item[]
  \item[] $Enc(a)$
  \item[] $Enc(x)$
\end{itemize}
\vspace{6mm}
\begin{itemize} \itemsep1pt \parskip0pt \parsep0pt
  \item[] Random integer $z$
  \item[] $Enc(a)^z, Enc(x)^z$
  \item[] $Enc(za), Enc(zx)$
\end{itemize}
\end{column}
\end{columns}

\end{frame}

%------------------------------------------------
%------------------------------------------------
\begin{frame}
\frametitle{PPWAP}

\begin{columns}
\begin{column}{5cm}
\hspace{16mm}{Alice}
\begin{itemize} \itemsep1pt \parskip0pt \parsep0pt
  \item[] 
  \item[] $\;$\newline
  \item[$3.$] \hspace{4mm}{\scriptsize $Enc(za+zb)$}
  \item[] \hspace{4mm}{\scriptsize $Enc(zx+zy)$}
\end{itemize}

\begin{itemize} \itemsep1pt \parskip0pt \parsep0pt
  \item[] \noindent\rule{11cm}{0.4pt}
  \item[$4.$] {\scriptsize $Decryption(Enc(za+zb), SK): za+zb$}
  \item[] {\scriptsize $Decryption(Enc(zx+zy), SK): zx+zy$}
\end{itemize}
\end{column}
\begin{column}{1cm}
\begin{itemize} \itemsep1pt \parskip0pt \parsep0pt
  \item[] \vspace{16 mm}
  \item[] \red{\Huge $\Leftarrow $}
  \item[] \vspace{8 mm}
  \item[] \red{\Huge $\Rightarrow $}
\end{itemize}
\vspace{1 cm}
\end{column}
\begin{column}{6.2cm}
\vspace{4 mm}
\hspace{16mm}Bob
\begin{itemize} \itemsep0pt \parskip0pt \parsep0pt
  \item[] {\scriptsize $Encryption(b ,PK):Enc(b)\Rightarrow Enc(zb)$}
  \item[] {\scriptsize $Encryption(y ,PK):Enc(y)\Rightarrow~Enc(zy)$}
  \item[] {\scriptsize $Enc(za+zb) = Enc(za)$}
  \item[] {\scriptsize $Enc(zb) Enc(zx+zy) = Enc(zx) Enc(zy)$}
\end{itemize}
\vspace{6mm}
\begin{itemize} \itemsep1pt \parskip0pt \parsep0pt
  \item[] 
  \item[] 
  \item[] 
\end{itemize}
\end{column}
\end{columns}

\centerline{$\frac{zx+zy}{za+zb} = \frac{x+y}{a+b}$ $\quad \Rightarrow \quad$ $\frac{x+y}{a+b}$}
\vspace{4 mm}
\end{frame}

%------------------------------------------------Enc(zb) Enc(zx+zy) = Enc(zx) Enc(zy)
%------------------------------------------------
\begin{frame}
\frametitle{Algorithm}

\begin{figure}[H]
  \centering
  \includegraphics[width=0.5\textwidth]{Algorithm}
  \caption{Two-party Jointly Decision Tree Algorithm.}
\end{figure}



\end{frame}
%------------------------------------------------
%------------------------------------------------
\begin{frame}
\frametitle{Result}

\begin{itemize} \itemsep10pt \parskip0pt \parsep0pt
  \item[$\bullet$] Demo
  \item[$\bullet$] Efficiency: The runtime depend on 3 factors.
  \begin{itemize} \itemsep2pt \parskip0pt \parsep0pt
  \item[$\ast$] Dataset size
  \item[$\ast$] Length of Key in encryption algorithm
  \item[$\ast$] Number of parties
\end{itemize}
\end{itemize}
\end{frame}

%------------------------------------------------
%------------------------------------------------
\begin{frame}
\frametitle{Algorithm Implement}
\begin{figure}[H]
  \centering
  \includegraphics[width=0.5\textwidth]{Welcome}
  \caption{Welcome Graphical User Interface.}
\end{figure}
\end{frame}
%------------------------------------------------
%------------------------------------------------
\begin{frame}
\frametitle{Algorithm Implement}
\begin{figure}[H]
  \centering
  \includegraphics[width=0.5\textwidth]{./SelfRun}
  \caption{Result of single-party ID3 algorithm on tic-tac-toe2 dataset.}
  \label{SelfRun}
\end{figure}

\end{frame}
%------------------------------------------------
%------------------------------------------------
\begin{frame}
\frametitle{Conclusion}

\begin{itemize} \itemsep10pt \parskip0pt \parsep0pt
  \item[$\bullet$] The PPWAP scheme is purposed in 2005 in PP K-means.
  \begin{itemize} \itemsep2pt \parskip0pt \parsep0pt
    \item[$\ast$] PPWAP can be extend to multi-party, supports Multi-party distributed ID3 algorithm.
  \end{itemize}

  \item[$\bullet$] Further research focus on improving the security level.
  \begin{itemize} \itemsep2pt \parskip0pt \parsep0pt
    \item[$\ast$] The scheme became safer and more complex.
  \end{itemize}

  \item[$\bullet$] Current research  focus on preventing malicious attack.
\end{itemize}
\end{frame}
%------------------------------------------------
%------------------------------------------------
\begin{frame}
\frametitle{Conclusion}

\begin{itemize} \itemsep5pt \parskip0pt \parsep0pt
  \item[$\bullet$] Select two large primes, $p$ and $q$. 
  \item[$\bullet$] Calculate the product $n=p\times q$, such that $gcd(n,\Phi(n)) = 1$, where $\Phi(n)$ is $(p-1)(q-1)$.
  \item[$\bullet$] Choose a random number $g$, where $g$ has order multiple of $n$ or $gcd(L(g^\lambda mod \; n^2 ),n) = 1$, where $L(t)= (t-1) / n$ and $\lambda(n)=lcm(p-1, q-1)$.
  \item[$\bullet$] The public key is composed of $(g, n)$, while the private key is composed of $(p,q,\lambda)$. 
  \item[$\bullet$] The Encryption of a message $m < n$ is given by: 
  \begin{itemize} \itemsep10pt \parskip0pt \parsep0pt
    \item[$\bullet$] $c=g^m\cdot r^n mod\; n^2$
  \end{itemize}
  \item[$\bullet$] The Decryption of ciphertext $c$ is given by: The Decryption of ciphertext $c$ is given by: 
  \begin{itemize} \itemsep10pt \parskip0pt \parsep0pt
    \item[$\bullet$] $m=(L(g^{\lambda} mod\; n^2 )/L(g^{\lambda} mod\; n^2 ) )mod \; n$
  \end{itemize}
\end{itemize}

\end{frame}
%------------------------------------------------





% %------------------------------------------------
% \begin{frame}
% \frametitle{Two-party jointly decision tree construction Example}
% \begin{figure}[H]
%   \centering
%   \includegraphics[width=0.8\textwidth]{Example2}
%   \caption{Decision tree construction result based on single-party dataset}
% \end{figure}
% \end{frame}

% %------------------------------------------------
% %------------------------------------------------
% \begin{frame}
% \frametitle{Two-party jointly decision tree construction Example}
% \begin{figure}[H]
%   \centering
%   \includegraphics[width=0.8\textwidth]{Example3}
%   \caption{Example for Playgolf Dataset Information Gain Calculation}
% \end{figure}
% \end{frame}

% %------------------------------------------------
%------------------------------------------------

% \begin{frame}
% \frametitle{Two-party jointly decision tree construction}

% \red{$D$} represent the dataset of Angelina, \red{$Attribute$} is the attributes of \red{$D$}, \red{$transInfo$} is the number of objects of each class in Bob dataset, and \red{$T$} is the corresponding target label of each object in dataset \red{$D$}. \red{$\widetilde{A}$} is the index of the best spliting attribute. We also provide a more detailed explanation of the whole algorithm for your further references(see \href{http://monaen.github.io/PrivacyDecisionTree/}{\magenta{Two-party jointly build decision tree}}). 

% \end{frame}

% %------------------------------------------------
% \begin{frame}
% \frametitle{Algorithm Implement}
% \begin{figure}[H]
%   \centering
%   \includegraphics[width=0.5\textwidth]{Welcome}
%   \caption{Welcome Graphical User Interface.}
% \end{figure}
% \end{frame}
% %------------------------------------------------
% %------------------------------------------------
% \begin{frame}
% \frametitle{Algorithm Implement}
% 1. load the dataset by clicking on the {\bf Load Data} button, and after you load the data, you can easily see the data you have loaded by clicking on the {\bf View Data} button (if you haven't loaded the data, you cannot see any result but a black table). \newline\newline
% 2. After you load the data you can see the result of ID3 algorithm by clicking on the button {\bf SelfRun}, and just like its name, the {\bf SelfRun} button implements the ID3 algorithm only based on the dataset of corresponding single party. 
% \newline\newline
% 3. After you load the data for both sides, then you can run the two-party jointly decision tree algorithm by clicking on button {\bf CombinedRun}, or you will see a warning on the Main Pane saying ``No data Loaded ...''. 
% \end{frame}
% %------------------------------------------------
% %------------------------------------------------
% \begin{frame}
% \frametitle{Algorithm Implement}
% 4. After you clicking on the button {\bf CombinedRun}, you will find that both the ``Main Pane'' and ``Mutual Information Pane'' print out the result.
% \newline\newline
% 5. The result on ``Main Pane'' is the final decision tree constructed using two-party jointly decision tree algorithm while the information printed on ``Mutual Information Pane'' is the transmit information from the other party. That is if you click on the {\bf CombinedRun} button of Angelina side, the printed information on ``Mutual Information Pane'' is the transmit information from Bob, while if you click on the {\bf CombinedRun} button of Bob side, the ``Mutual Information Pane'' shows the transmit information from Angelina.
% \end{frame}
% %------------------------------------------------
% %------------------------------------------------
% \begin{frame}
% \frametitle{Algorithm Implement}
% \begin{figure}[H]
%   \centering
%   \includegraphics[width=0.5\textwidth]{./SelfRun}
%   \caption{Result of single-party ID3 algorithm on tic-tac-toe2 dataset.}
%   \label{SelfRun}
% \end{figure}

% \end{frame}
% %------------------------------------------------







\begin{frame}
\Huge{\centerline{The End}}
\end{frame}

%----------------------------------------------------------------------------------------

\end{document}